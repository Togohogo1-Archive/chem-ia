\documentclass[12pt]{article}

\usepackage[table]{xcolor}
\usepackage[utf8]{inputenc}
\usepackage[style=apa]{biblatex}
\usepackage[margin=1in]{geometry}
% \usepackage[indent=0.5in]{parskip}  % remember to add comment about why this needed (it also has some issues)
\usepackage{hyperref}
\usepackage{indentfirst}
\usepackage{mhchem}
\usepackage{enumitem}
\usepackage{xcolor}
\usepackage{mathtools}
\usepackage{array}
\usepackage{caption}
\usepackage{float}
\usepackage{tabularx}

\addbibresource{references.bib}
\setlength{\bibitemsep}{12pt}
\setlength{\parindent}{0in}
% \setlength{\parskip}{12pt}
\captionsetup[table]{skip=6pt}
\renewcommand{\baselinestretch}{1.15}  % 1.15 spacing

\title{\textbf{The Relationship between Glucose Concentration and the Rate of Carbon Dioxide Production during the Fermentation Process of Yeast}}
\author{}
\date{}

% https://chem-space.com/search
\begin{document}

\pagenumbering{gobble}
\maketitle
\newpage
\pagenumbering{arabic}

\section{Introduction}
I have always enjoyed cooking as a mostly self-taught hobby. Having experimented with multiple aspects of cooking throughout my life eventually led me to explore baking, where I was fascinated by timelapse footages of bread rising in the oven. As such, I was inspired to make my own bread, following basic recipes online with a surface-level understanding of the chemistry involving the release of carbon dioxide from yeast fermentation. However, the realm of baking is very large and not limited to bread only, as there exist many types of doughs for pizza, puff pastry, cookies, etc. One notable difference between these different types of dough is the amount of sugar used, which led me to form my research question: \textbf{``The Relationship between Glucose Concentration and the Rate of Carbon Dioxide Production during the Fermentation Process of Yeast''}.

\section{Background}

\subsection{What Is Yeast}
Yeast are single celled, eukaryotic organisms of the fungus kingdom. Like many living organisms, yeast requires food to survive, namely be converting carbohydrates or sugar sources to energy through either aerobic or anaerobic respiration \parencite{ref}. There exists many types of yeast. The one used in this internal assessment are part of the \emph{Saccharomyces cerevisiae} strain, which can be categorized into brewer's yeast (beer making) and baker's yeast (leavening agent) \parencite{ref}.

\medskip

This internal assessment focuses on the use of active dry yeast, which consists of pellets of live yeast cells coated with a layer of dehydrated cells, \parencite{ref} allowing it to stay dormant to give it a longer shelf life. Hence, active dry yeast must be activated through rehydration in lukewarm water from 37$\,^{\circ}$C to 43$\,^{\circ}$C (optimal fermentation conditions) for 5 to 10 minutes in a process called proofing which also ensures that the yeast is dissolved in the water \parencite{ref}.

\subsection{Aerobic and Anaerobic Respiration in Yeast}
It is well known that yeast reacts with glucose to produce ethanol and carbon dioxide. To understand the details of how this works, it is crucial to understand how yeast undergoes both aerobic and anaerobic respiration.

\medskip

Aerobic respiration is much more efficient and optimal than anaerobic respiration because of the significantly more ATP (adenosine triphosphate, energy carrying molecule) produced. The chemical formula for aerobic respiration is given below:
\begin{equation}
    \ce{\underset{\text{glucose}}{\ce{C6H12O6}} + 6O2 ->[cellular respiration] 6CO2 + 6H2O + ATP}
\end{equation}
Going into the details of aerobic respiration its three steps:
\begin{enumerate}[topsep=\parskip, noitemsep]
    \item Glycolysis
    \item Tricarboxylic acid cycle
    \item Oxidative phosphorylation (where most of the APT is produced)
\end{enumerate}

% TODO italicize key terms?

\medskip

However, the focus of this internal assessment is on anaerobic respiration of yeast because the amount of oxygen was limited given the environment that the experiment was carried out in. The only similarity between aerobic and anaerobic respiration is glycolysis, which prompts to explore this step in more detail \parencite{ref}.

\medskip

The first step of glycolysis converts glucose into fructose 1,6-biphosphate (F-1,6-BP), outlined in the figure below:

\medskip

[fig- Conversion of glucose to F-1,6-BP. Intermediate steps are ommited. Image created by moi]

\medskip

Glucose goes through a series of reactions catalyzed by multiple enzymes, which consumes 2 ATP in the process. ATP turns to ADP (adenosine diphosphate) when a phosphate group (\ce{HPO_{4}^{2-}}) gets transferred from the ATP to the sugars that glucose gets broken down into, eventually leading to the formation of F-1,6-BP.

\medskip

The second step of glycolysis involves splitting F-1,6-BP into 2 three-carbon fragments, glyceraldehyde 3-phosphate (GAP) and dihydroxyacetone phosphate (DHAP), outlined in the figure below:

\medskip

[fig]

\medskip

When catalyzed with an enzyme, F-1,6-BP splits into the three-carbon fragments but only GAP can continue the glycolysis process. However, GAP and DHAP exist in equilibrium, meaning all the DHAP eventually turns to GAP.

\medskip

The third and final step of glycolysis converts the DHAP to pyruvate, another three-carbon molecule while producing ATP, outlined in the figure below:

\medskip

[fig]

\medskip

Like glucose, GAP goes through another series of reactions catalyzed by multiple enzymes that eventually turn it into Pyruvate. During one step of the process, a molecule called \ce{NAD+} (nicotinamide adenine dinucleotide, a coenzyme found in the cell that is crucial to metabolic processes) \parencite{ref} gets reduced to NADH wihle GAP is simultaneously oxidized. The phosphate groups that were given from the ATP in the first step is combined with ADP twice to form 2 ATP. Since the DHAP got converted into GAP in the previous step, 4 total ATP is formed \parencite{ref}.

\medskip

The net equation for the formation of pyruvate is thus given by:
\begin{multline}
    \ce{glucose + 2HPO_{4}^{2-} + 2ADP + 2NAD+ \\-> 2pyruvate + 2ATP + 2 NADH + 2H+ + 2H2O}
\end{multline} % TODO \ce{2pyruvate} issue
According to Equation (2), the end result of glycolysis leaves 2 pyruvate, 2 NADH, 2 ATP, 2 \ce{H2O}, and 2 \ce{H+}. If there is unreacted glucose, glycolysis will continue. However, a cell's supply of \ce{NAD+} is limited, so it must be regenerated. In aerobic respiration, the pyruvate goes through the tricarboxylic acid cycle to produce even more NADH that eventually gets oxidized to \ce{NAD+} during oxidative phosphorylation, which produces most of the ATP associated with aerobic respiration as a bonus. In anaerobic respiration, fermentation provides an alternate pathway to turn the NADH back to \ce{NAD+}.

\medskip

% TODO figure out when to use name and chemical
% TODO chemical capitals
% TODO add more references
For yeast,  alcoholic fermentation converts pyruvate into ethanol and carbon dioxide, both of with are waste products. It consists of two steps that are also catalyzed by enzymes. The first step in this conversion process involves the decarboxylation of pyruvate to acetaldehyde, producing carbon dioxide as a by product, as shown below:
\begin{equation}
    \ce{\underset{\text{pyruvate}}{\ce{CH3COCOO-}} + H+ -> \underset{\text{acetaldehyde}}{\ce{CH3CHO}} + CO2}
\end{equation}
In the second step, acetaldehyde gets reduced to ethanol and NADH gets oxidized to \ce{NAD+}, effectively regenerating \ce{NAD+} to be reused in glycolysis:
\begin{equation}
    \ce{CH3CHO + NADH + H+ -> \underset{\text{ethanol}}{\ce{C2H5OH}} + NAD}
\end{equation}
Equation (3) can be substituted in Equation (2) after some rearranging:
\begin{multline*}
    \ce{glucose + 2HPO_{4}^{2-} + 2ADP + 2NAD+ \\-> \textcolor{red}{\ce{2CH3CHO}} + \textcolor{red}{\ce{2CO2}} + 2ATP + 2NADH + 2H2O}
\end{multline*}
Adding \ce{2H+} on both sides and substituting Equation (4)
\begin{multline*}
    \ce{glucose + 2HPO_{4}^{2-} + 2ADP + 2NAD+ + \textcolor{blue}{\ce{2H+}} \\-> \textcolor{blue}{\ce{2C2H5OH}} + \textcolor{blue}{\ce{2NAD+}} + 2ATP + 2H2O}
\end{multline*}
Cancelling \ce{2NAD+} on both sides gives:
\begin{equation}
    \ce{C6H12O6 + 2HPO_{4}^{2-} + 2ADP + 2H+ -> 2C2H5OH + 2CO2 + 2ATP + 2H2O}
\end{equation}
Which is more familiarly known as:
\begin{equation}
    \ce{C6H12O6 ->[yeast] 2C2H5OH + 2CO2}
\end{equation}

\section{Kinetics}  % TODO fermentation process changen to alcoholic fermentation

\subsection{Basic Kinetics}
Given the following chemical reaction denoting alcoholic fermentation:
\begin{equation}
    \ce{A -> 2B + 2C}
\end{equation}
Studying the kinetics of this reaction is useful for performing further levels of analysis. Kinetics in chemistry refers to the study of rates of chemical reactions, which is a measurement of the change in concentration of products and reactants over a period of time \parencite{ref}.

\medskip

The rate of reaction, also known as the reaction velocity $V$ with respect to a compound within the chemical equation is represented as the change in concentration of that compound over the change in time multiplied by the inverse of its stoichiometric coefficient:
\begin{equation}
    V = \text{rate} = -\frac{d[A]}{dt} = \frac{1}{2}\frac{d[B]}{dt} = \frac{1}{2}\frac{d[C]}{dt}
\end{equation}

Additionally, the rate law states how the reaction velocity will be proportional to the product of reactants, each reactant raised to a power:
\begin{equation}
    V = k[A]^m
\end{equation}
where k is is the proportionality constant \parencite{ref} and the sum of the powers denote the order of the reaction. In the case of alcoholic fermentation, the order is purely dependent on $m$.

\medskip

If the order of the reaction is 0, the rate of reaction will not be dependent on the concentration of the reactants, that is $V = k[A]^0 = k$. If the order of the reaction is 1, the rate of reaction will be proportional to the concentration of the reactants, that is $V = k[A]^1 = k[A]$.

\subsection{Michaelis Menten Enzyme Kinetics}

\section{Design}

\newpage

\subsection{Safety}
Before starting the lab, physical, environmental, and ethical safety issues must be taken into account. The following table lists them out:
\begin{table}[H]
\centering
\caption{Safety hazards and their considerations}
\label{table:1}
\begin{tabularx}{\textwidth} {
    | >{\hsize=.4\hsize \linewidth=\hsize \raggedright\arraybackslash}X
    | >{\hsize=.6\hsize \linewidth=\hsize \raggedright\arraybackslash}X |
}
    \hline
    \rowcolor[HTML]{CCCCCC} Hazard & Safety Considerations \\
    \hline
    Risk of broken glassware & If possible, glassware should be handled with both hands to minimize the chance of breakage. Extra care should be taken during the cleaning process as glassware may be slippery when wet.\\
    \hline
    Risk of hotplate burns & It is important to not touch the hotplate, even when it's off. Once finished using, the hotplate and any lab apparatus that was on it should be cooled before storage.\\
    \hline
    Wiring and electronics & Baggy clothing should not be worn due to the risk of accidentally tangling with wires. If not used in an experiment, chemicals should be kept a far distance from electronics.\\
    \hline
    Dumping products down the drain & Yeast should not be dumped down the drain unless it is dissolved and heavily diluted. Similarly, dumping concentrated ethanol is hazardous unless heavily diluted as well. \parencite{ref}\\
    \hline
    Wasting yeast & Yeast is used in baking and wasting it can be an ethical concern. Any unused yeast that hasn't been contaminated with a scoopula should be returned.\\
    \hline
    Spills and splashes & Goggles should be worn until the lab has been cleared up to prevent chemicals from coming in contact with the eyes. Paper towels should be kept handy at all times to clean up spills.\\
    \hline
\end{tabularx}
\end{table}

\subsection{Hypothesis}
After the yeast is fully activated and dissolved and mixed with glucose, the concentration of carbon dioxide produced will increase linearly for a short time until tapering off as all of the glucose gets used when the reaction goes to completion. For lower concentrations of glucose, the initial reaction velocity and the final concentration of carbon dioxide gas produced will be smaller than higher concentrations of glucose. When plotting the initial reaction velocity against the concentration of glucose, the curve of best-fit will fit the Michaelis-Menten kinetics model, passing through the origin.

\subsection{Variables}

\subsection{Apparatus and Experimental Setup}

\section{Data Collection and Processing}

\section{Analysis}

\section{Evaluation}

\section{Conclusion}

\nocite{*}

\newpage

\printbibliography

\newpage

% TODO figure out why "appendices" are not preferred
\section{Appendices}

\subsection{Appendix A}

\subsection{Appendix B}

\subsection{Appendix C}

\subsection{Appendix D}

\subsection{Appendix E}

\end{document}