\documentclass{article}

\usepackage[table]{xcolor}
\usepackage[utf8]{inputenc}
\usepackage[style=apa]{biblatex}
\usepackage[margin=1in]{geometry}
\usepackage{hyperref}
\usepackage{indentfirst}
\usepackage{mhchem}
\usepackage{enumitem}
\usepackage{mathtools}
\usepackage{array}
\usepackage{caption}
\usepackage{float}
\usepackage{tabularx}
\usepackage{multirow}
\usepackage{siunitx}
\usepackage{amssymb}
\usepackage{graphicx}
\usepackage{subcaption}

% TODO still haven't decided when to use co2 or concentration
\addbibresource{references.bib}
\setlength{\bibitemsep}{12pt}
\setlength{\parindent}{0in}
\captionsetup[table]{skip=6pt}
\renewcommand{\baselinestretch}{1.15}

\title{\textbf{The Relationship between Glucose Concentration and the Rate of Carbon Dioxide Production during the Fermentation Process of Yeast}}
\author{}
\date{}

\begin{document}

\pagenumbering{gobble}
\maketitle
\newpage
\pagenumbering{arabic}

\section{Introduction}
I have always enjoyed cooking as a mostly self-taught hobby. Having experimented with multiple aspects of cooking throughout my life eventually led me to explore baking, where I was fascinated by timelapse footages of bread rising in the oven. As such, I was inspired to make my own bread, finding and reading basic recipes with a surface-level understanding of the chemistry involving the release of carbon dioxide from yeast fermentation. However, the realm of baking is very large and not limited to bread only, as there exist many types of doughs for pizza, puff pastry, cookies, etc. One notable difference between these different types of dough is the amount of sugar used, and although some of these recipes do not call for yeast, I was still interested by its function in baking, which led me to form my research question: \textbf{``The Relationship between Glucose Concentration and the Rate of Carbon Dioxide Production during the Fermentation Process of Yeast''}.

\section{Background}

\subsection{What Is Yeast}
Yeast are single celled, eukaryotic organisms of the fungus kingdom. Like many other life forms, yeast requires food to survive, by converting carbohydrates or saccharides into energy through aerobic or anaerobic respiration (“What is Yeast?”, 2015). Among the many different types of yeast, the one used in this internal assessment is part of the \emph{Saccharomyces cerevisiae} strain - specifically, baker's yeast (The Editors
of Encyclopaedia Britannica, n.d.).

\medskip

This internal assessment will focus on the use of active dry yeast, which contain pellets of live yeast cells coated with a layer of dehydrated yeast cells, (Sherdon, 2016) allowing it to stay dormant, giving it a longer shelf life. Hence, active dry yeast must be activated through rehydration in lukewarm water from 37\,\si{\celsius} to 43\,\si{\celsius} (optimal fermentation conditions) for 5 to 10 minutes in a process called proofing, ensuring that the yeast gets dissolved in the water (Poilâne, 2022-02-24).

\subsection{Aerobic and Anaerobic Respiration in Yeast}
It is well known that yeast reacts with glucose to produce ethanol and carbon dioxide. To explore the details of how this works, it is crucial to understand how yeast undergoes both aerobic and anaerobic respiration.

\medskip

Aerobic respiration is much more efficient and optimal than anaerobic respiration because significantly more ATP (adenosine triphosphate, energy carrying molecule) is produced (Berg et al., 2006). The chemical formula for aerobic respiration is given below:
\begin{equation}
    \ce{\underset{\text{Glucose}}{\ce{C6H12O6}} + 6O2 ->[cellular respiration] 6CO2 + 6H2O + ATP}
\end{equation}

Aerobic respiration is broken down into 3 steps:
\begin{enumerate}[topsep=\parskip, noitemsep]
    \item Glycolysis
    \item Tricarboxylic acid cycle
    \item Oxidative phosphorylation (where most of the APT is produced)
\end{enumerate}

\medskip

However, the focus of this internal assessment is on anaerobic respiration of yeast due to the limited supply of oxygen in the environment that the experiment was carried out in. When exploring the details of anaerobic respiration, glycolysis is determined to be a step that both aerobic and anaerobic respiration share, and it itself can be split up ito 3 steps (Berg et al., 2006).

\newpage

The first step of glycolysis converts glucose into fructose 1,6-biphosphate (F-1,6-BP), shown in the figure below:
\begin{figure}[H]
    \centering
    \includegraphics[width=0.8\linewidth]{figures/figure_01.png}
    \caption{Conversion of glucose to F-1,6-BP. Intermediate steps are represented by the arrows.}
    \label{fig:figure1}
\end{figure}

Glucose goes through a series of reactions catalyzed by multiple enzymes (represented by straight arrows), which consumes \ce{2ATP} in the process. ATP turns to ADP (adenosine diphosphate) when a phosphate group (\ce{HPO_{4}^{2-}}) gets transferred from each ATP to the sugars that glucose gets broken down into, before forming F-1,6-BP (Berg et al., 2006).

\medskip

The second step of glycolysis involves splitting F-1,6-BP into 2 three-carbon fragments, glyceraldehyde 3-phosphate (GAP) and dihydroxyacetone phosphate (DHAP), shown in the figure below:
\begin{figure}[H]
    \centering
    \includegraphics[width=0.55\linewidth]{figures/figure_02.png}
    \caption{Conversion of F-1,6-BP to GAP and DHAP}
    \label{fig:figure2}
\end{figure}

Only GAP can continue the glycolysis process, but since GAP and DHAP exist in equilibrium, all the DHAP eventually gets converted to GAP (Berg et al., 2006).

\newpage

The third and final step of glycolysis converts the DHAP to pyruvate, another three-carbon molecule, shown in the figure below:
\begin{figure}[H]
    \centering
    \includegraphics[width=0.886\linewidth]{figures/figure_03.png}
    \caption{Conversion of DHAP to pyruvate, which happens twice. Intermediate steps are represented by the arrows.}
    \label{fig:figure3}
\end{figure}

Like glucose, GAP goes through another series of reactions catalyzed by multiple enzymes that eventually turn it into Pyruvate. During one step of the process, a molecule called \ce{NAD+} (nicotinamide adenine dinucleotide, a coenzyme found in the cell that is crucial to metabolic processes)  (“What is NAD+ and Why
Is it Important?”, n.d.) gets reduced to NADH while GAP is simultaneously oxidized. The phosphate groups that were taken from ATP in the first step is combined with ADP twice to form \ce{2ATP}. Since the DHAP got converted into GAP in the previous step, 4 total ATP is formed (Berg et al., 2006). Thus, the net equation for the formation of pyruvate is thus given by:
\begin{equation}
    \ce{Glucose + 2HPO_{4}^{2-} + 2ADP + 2NAD+ -> 2Pyruvate + 2ATP + 2 NADH + 2H+ + 2H2O}
\end{equation}

According to Equation (2), the end result of glycolysis leaves \ce{2pyruvate}, \ce{2NADH}, \ce{2ATP}, \ce{2H2O}, and \ce{2H+}. If there is unreacted glucose, glycolysis will continue. However, a cell's supply of \ce{NAD+} is limited, so it must be regenerated. In aerobic respiration, the pyruvate goes through the tricarboxylic acid cycle to produce even more NADH that eventually gets oxidized to \ce{NAD+} during oxidative phosphorylation. In anaerobic respiration, fermentation provides an alternate pathway to turn the NADH back to \ce{NAD+}.

\medskip

% TODO figure out when to use name and chemical
% TODO chemical capitals
% TODO add more references
For yeast, alcoholic fermentation converts pyruvate into ethanol and carbon dioxide, both of with are waste products. It consists of 2 steps that are also catalyzed by enzymes. The first step in this conversion process involves the decarboxylation of pyruvate to acetaldehyde, producing carbon dioxide as a by product, as shown below (Berg et al., 2006):
\begin{equation}
    \ce{\underset{\text{Pyruvate}}{\ce{CH3COCOO-}} + H+ -> \underset{\text{Acetaldehyde}}{\ce{CH3CHO}} + CO2}
\end{equation}

In the second step, acetaldehyde gets reduced to ethanol and NADH gets oxidized to \ce{NAD+}, effectively regenerating \ce{NAD+} to be reused in glycolysis:
\begin{equation}
    \ce{CH3CHO + NADH + H+ -> \underset{\text{Ethanol}}{\ce{C2H5OH}} + \ce{NAD+}}
\end{equation}

Given the previous few equations, some rearranging can be done to derive the full equation that represents alcoholic fermentation. Substituing Equation (3) in Equation (2):
\begin{equation*}
    \ce{Glucose + 2HPO_{4}^{2-} + 2ADP + 2NAD+ -> \textcolor{red}{\ce{2CH3CHO}} + \textcolor{red}{\ce{2CO2}} + 2ATP + 2NADH + 2H2O}
\end{equation*}

Adding \ce{2H+} on both sides and substituting Equation (4)
\begin{equation*}
    \ce{Glucose + 2HPO_{4}^{2-} + 2ADP + 2NAD+ + \textcolor{blue}{\ce{2H+}} -> \textcolor{blue}{\ce{2C2H5OH}} + \textcolor{blue}{\ce{2NAD+}} + 2ATP + 2H2O}
\end{equation*}

Cancelling \ce{2NAD+} on both sides gives:
\begin{equation}
    \ce{C6H12O6 + 2HPO_{4}^{2-} + 2ADP + 2H+ -> 2C2H5OH + 2CO2 + 2ATP + 2H2O}
\end{equation}

Which is more familiarly known as:
\begin{equation}
    \ce{C6H12O6 ->[yeast] 2C2H5OH + 2CO2}
\end{equation}

\section{Kinetics}

\subsection{Basic Kinetics}
Given the following chemical reaction denoting alcoholic fermentation:
\begin{equation}
    \ce{A -> 2B + 2C}
\end{equation}
Studying the kinetics of this reaction is useful for predicting its behvaiour. Kinetics in chemistry refers to the study of rates of chemical reactions, which is a measurement of the change in concentration of products and reactants over a period of time.

\medskip

The rate of reaction, also known as the reaction velocity $V$ with respect to a compound within the chemical equation, is represented as the change in concentration of that compound over the change in time multiplied by the inverse of its stoichiometric coefficient:
\begin{equation}
    V = \text{rate of reaction} = -\frac{d[A]}{dt} = \frac{1}{2}\frac{d[B]}{dt} = \frac{1}{2}\frac{d[C]}{dt}
\end{equation}

Additionally, the rate law states how the reaction velocity will be directly proportional to the product of reactants, each reactant raised to a power:
\begin{equation}
    V = k[A]^m
\end{equation}
Where $k$ is is the rate constant and the sum of the powers denote the order of the reaction. In the case of alcoholic fermentation, the order is purely dependent on $m$.

\medskip

If the order of the reaction is 0, the rate of reaction will not be dependent on the concentration of the reactants, that is $V = k[A]^0 = k$. If the order of the reaction is 1, the rate of reaction will be proportional to the concentration of the reactants, that is $V = k[A]^1 = k[A]$.

\subsection{Michaelis Menten Enzyme Kinetics}
When reactions are enzyme-catalyzed, enzyme kinetics may also be used to predicting the behvaiour of the reaction. The Michaelis Menten model, proposed by Leonor Michaelis and Maud Menten in 1913, is concerned with the relationship between initial reaction velocity $V_0$ (rate of enzyme catalysis), and the substrate concentration (Berg et al., 2006):
\begin{equation}
    \ce{E + S <=>[k_1][k_{-1}] ES ->[k_2] E + P}
\end{equation}

Where
\begin{itemize}[topsep=\parskip, noitemsep]
    \item E denotes the enzyme zyamse, the main enzyme responsible for fermentation in yeast (“Enzymes - Understanding Ingredients for the Canadian Baker”, 2015)
    \item S denotes the substrate, the glucose solution
    \item P denotes the product, the carbon dioxide gas formed
    \item ES denotes the enzyme substrate complex, an intermediate step in where the enzyme temporarily binds to the substrate, which eventually gets dissociated into the same enzyme and the product
\end{itemize}

\medskip

Additionally:
\begin{itemize}[topsep=\parskip, noitemsep]
    \item $k_1$ denotes the rate constant associated with the formation of \ce{ES}
    \item $k_{-1}$ is associated with the breakdown of \ce{ES} to \ce{E + S}
    \item $k_2$ is associated with the dissociation of \ce{ES} to \ce{E + P}
\end{itemize}

\medskip

Technically, $k_{-2}$ also exists, but since the Michaelis Menten model is only concerned with $V_0$, the breakdown of \ce{E + P} into \ce{ES} is negligible.

\medskip

Firstly, a formula can be derived for $V_0$, using the equation \ce{ES -> E + P}:
\begin{equation}
    V_0 = k_2[\ce{ES}]
\end{equation}

The rate of formation of \ce{ES} is given by:
\begin{equation}
    k_1[\ce{E}][\ce{S}]
\end{equation}

And the rate of breakdown of \ce{ES} is given by:
\begin{equation}
    (k_{-1} + k_2) [\ce{ES}]
\end{equation}

In 1924, George Briggs and John Haldane proposed the steady state assumption to further simplify enzyme kinetics, which assumes that the rate of formation and the rate of breakdown of \ce{ES} are equal (Berg et al., 2006):
\begin{equation}
    k_1[\ce{E}][\ce{S}] = (k_{-1} + k_2) [\ce{ES}]
\end{equation}

Rearranging Equation (14) gives:
\begin{equation}
    \frac{[\ce{E}][\ce{S}]}{[\ce{ES}]} = \frac{k_{-1}+k_2}{k_1}
\end{equation}

$\frac{k_{-1}+k_2}{k_1}$ can be defined as a new variable $K_m$, the Michaelis constant. Additionally, the equation can be rearranged to represent [\ce{ES}] in terms of concentrations of known substances:
\begin{equation}
    [\ce{ES}] = \frac{[\ce{E}][\ce{S}]}{K_m}
\end{equation}

Considering the total amount of enzyme in the reaction as $\ce{E_T}$, consisting of enzymes bounded to the substrate and free enzymes. [\ce{E_T}] can then be represented as follows:
\begin{equation}
    [\ce{E_T}] = [\ce{E}] + [\ce{ES}]
\end{equation}

Solving Equation (17) for [\ce{E}] and substituting it in Equation (16) gives:
\begin{equation}
    [\ce{ES}] = \frac{([\ce{E_T}] - [\ce{ES}]) [\ce{S}]}{K_m}
\end{equation}

Rearranging Equation (18) to represent [\ce{ES}] in terms of known values:
\begin{equation}
    [\ce{ES}] = [\ce{E_T}]\frac{[\ce{S}]}{[\ce{S}] + K_m}
\end{equation}

Taking Equation (11) and substituting [\ce{ES}]:
\begin{equation}
    V_0 = k_2[\ce{E_T}]\frac{[\ce{S}]}{[\ce{S}] + K_m}
\end{equation}

$k_2[\ce{E_T}]$ can also be used to describe the reaction velocity. If [\ce{E_t}] was used in place of [\ce{ES}] in Equation (11), all enzymes are undergoing catalysis. This state defined as $V_{max}$, the maximum reaction velocity for a set of conditions. Rewriting Equation (20) gives the final Michaelis Menten equation (Berg et al., 2006):
\begin{equation}
    V_0 = V_{max}\frac{[S]}{[S] + K_m}
\end{equation}

If $[\ce{S}] << K_m$, then $V_0 \approx V_{max}\frac{[\ce{S}]}{K_m}$, meaning that the reaction velocity will be directly proportional to the substrate concentration (first order reaction). If $[\ce{S}] >> K_m$, then $V_0 \approx V_{max}$, meaning the reaction velocity does not depend on the substrate concentration (zero order reaction). If $[\ce{S}] = K_m$, then $V_0 = \frac{V_{max}}{2}$, meaning $K_m$ always equals the substrate concentration at half the maximum reaction velocity. This information makes intuitive sense when graphed:
\begin{figure}[H]
    \renewcommand{\figurename}{Graph}
    \setcounter{figure}{0}
    \centering
    \includegraphics{figures/graph_01.png}
    \vspace*{-12pt}
    \caption{The Michaelis Menten model with $V_{max} = 1$ and $K_m = 1$}
    \label{fig:graph1}
\end{figure}

\newpage

\section{Design}

\subsection{Safety}
Before starting the lab, physical, environmental, and ethical safety issues must be taken into account. The following table lists them out:
\begin{table}[H]
\centering
\caption{Safety hazards and their considerations}
\label{table:1}
\begin{tabularx}{\textwidth} {
    | >{\hsize=.4\hsize \linewidth=\hsize \raggedright\arraybackslash}X
    | >{\hsize=.6\hsize \linewidth=\hsize \raggedright\arraybackslash}X |
}
    \hline
    \textbf{Hazard} & \textbf{Safety Considerations} \\
    \hline
    Risk of broken glassware & If possible, glassware should be handled with both hands to minimize the chance of breakage. Extra care should be taken during the cleaning process as glassware may be slippery when wet.\\
    \hline
    Risk of hotplate burns & It is important to not touch the hotplate, even when it is off. Once finished using, the hotplate and any lab apparatus that came in contact with it should be cooled before storing.\\
    \hline
    Wiring and electronics & Baggy clothing should not be worn to avoid the risk of accidentally tangling with wires. If not used in an experiment, chemicals should be kept far away from electronics.\\
    \hline
    Dumping products down the drain & Yeast should not be dumped down the drain unless it is dissolved and heavily diluted (Holmes, n.d.). Similarly, dumping concentrated ethanol is hazardous unless heavily diluted as well (“Sewer Disposal: What Can Go Down the Drain?”, 2021). \\
    \hline
    Wasting yeast & Yeast used in this experiment is also used in baking, and wasting it could be an ethical concern. Any unused yeast that has not been contaminated with a scoopula should be returned.\\
    \hline
    Spills and splashes & PPE such as goggles should be worn until the lab is cleared to prevent chemicals from coming in contact with the eyes. Paper towels should be available at all times to clean up spills.\\
    \hline
\end{tabularx}
\end{table}

\subsection{Hypothesis}
Given that the yeast was fully activated and dissolved, the concentration of carbon dioxide produced will increase linearly for a short time after glucose is added, until tapering off, such that the rate of carbon dioxide production reaches zero as the reaction reaches equilibrium. For lower concentrations of glucose, the initial reaction velocity and the final concentration of carbon dioxide gas produced will be smaller than higher concentrations of glucose. When plotting the initial reaction velocity against various concentrations of glucose, the curve of best fit will fit the Michaelis Menten kinetics model, passing through the origin.

\newpage

% todo table below or following table
\subsection{Variables}
The following table lists out all dependent, independent, and control variables used in the experiment, along with reasons and details to why each variable was chosen to be a specific type:
\begin{table}[H]
\centering
\caption{Variables used in the experiment}
\label{table:2}
\begin{tabularx}{\textwidth} {
    | >{\hsize=.2\hsize \linewidth=\hsize \raggedright\arraybackslash}X
    | >{\hsize=.3\hsize \linewidth=\hsize \raggedright\arraybackslash}X
    | >{\hsize=.5\hsize \linewidth=\hsize \raggedright\arraybackslash}X |}
    \hline
    \textbf{Variable type} & \textbf{Description} & \textbf{Reason and details} \\
    \hline
    \multirow[t]{2}{\hsize}{Independent} & Time (\si{s}) & Time is an independent variable when plotting a concentration-time graph. \\
    \cline{2-3}
    % TODO more consistent use of \si for all units
    & Glucose concentration (\si{M}) & The glucose concentration will be varied by modifying the mass of glucose. \\
    \hline
    \multirow[t]{2}{\hsize}{Dependent} & Carbon dioxide concentration (\si{ppm}) & When the reaction occurs, the carbon dioxide concentration will vary with time. Specific units for ppm are \si{mL.L^{-1}}. \\
    \cline{2-3}
    & Initial rate of carbon dioxide production (\si{ppm.s^{-1}}) & The initial rate of carbon dioxide production will depend on the glucose concentration. \\
    \hline
    \multirow[t]{3}{\hsize}{Control} & Volume of deionized water (\si{mL}) & Deionized water will be used instead of tap water since disassociated ions in tap water may influence rate of reaction. \\
    \cline{2-3}
    & Temperature of water (\si{\celsius}) and mass of yeast used (\si{g}) & These variables will be kept constant because they both influence the rate of carbon dioxide production, and the effects of modifying only one independent variable is studied. \\
    \cline{2-3}
    & Time for the yeast to fully dissolve and activate & The yeast will be given ample time to dissolve and activate in the deionized water. If the glucose is added too early, then the initial rate of reaction will be smaller than expected. \\
    \hline
\end{tabularx}
\end{table}

\newpage

\subsection{Apparatus and Experimental Setup}
The following materials were used in the experiment:
\begin{table}[H]
\centering
\caption{Categorized materials list}
\label{table:3}
\begin{tabularx}{\textwidth} {
    | >{\hsize=.35\hsize \linewidth=\hsize \raggedright\arraybackslash}X
    | >{\hsize=.65\hsize \linewidth=\hsize \raggedright\arraybackslash}X |}
    \hline
    \textbf{Category} & \textbf{Materials} \\
    \hline
    Compounds and reactants & \textbullet Red Star active dry yeast \\
    & \textbullet Lab grade glucose \\
    & \textbullet Deionized water \\
    \hline
    Containers and tools & \textbullet 1\,\si{L} beaker \\
    & \textbullet 100\,\si{mL} graduated cylinder $(\pm0.5\,\si{mL})$ \\  % TODO turn brackets into $()$
    & \textbullet 250\,\si{mL} Nalgene bottle \\
    & \textbullet Magnetic stir bar \\
    & \textbullet Folded paper for depositing reactants when weighing \\
    & \textbullet Scoopula \\
    \hline
    Devices & \textbullet Barnstead/Thermolyne magnetic stirring hotplate \\
    & \textbullet Vernier LabQuest2 \\
    & \textbullet Vernier stainless steel temperature probe \\
    & \textbullet A\&D EJ-200 scale $(\pm0.01\,\si{g})$ \\
    & \textbullet Vernier \ce{CO2} gas sensor \\
    \hline
\end{tabularx}
\end{table}

The experiment was carried out following the procedure below:
\begin{enumerate}[topsep=\parskip, noitemsep]
    % \setlength\itemsep{-5pt}
    \item Approximately 0.5\,\si{L} of deionized water was prepared in a 1\,\si{L} beaker and placed on a magnetic stirring hotplate.
    \item The Vernier LabQuest was plugged in with a Vernier temperature probe.
    \item The water was heated until it reached approximately 37\,\si{\celsius}. The temperature of the hotplate was kept at around 40\,\si{\celsius} to 46\,\si{\celsius} to account for cooling.
    \item 0.10\,\si{g} of glucose and 2.00\,\si{g} of Red Star active dry yeast were weighed
    \item The yeast was deposited in the 250\,\si{mL} Nalgene bottle with the magnetic stir bar.
    \item 100\,\si{mL} of water was measured with a graduated cylinder was poured from the 1\,\si{L} beaker to the bottle.
    \item After screwing the cap on, the bottle transferred to the warm water bath.
    \item The Vernier \ce{CO2} gas sensor was connected and calibrated to read high concentrations of \ce{CO2} gas. The LabQuest was set to 20 seconds per sample for 10 minutes.
    \item Magnetic stirring was activated for about 10 minutes to mix the yeast solution, ensuring that all the yeast was dissolved and activated. During this phase, observations were recorded and the \ce{CO2} gas sensor was warmed up to read a baseline \ce{CO2} concentration.
    \item After the yeast was activated, glucose was deposited in the bottle and the cap was immediately swapped with the \ce{CO2} gas sensor to prevent \ce{CO2} from escaping.
    \item The Nalgene-\ce{CO2} sensor setup was lightly swirled around the beaker, allowing the stir bar to reach all corners of the bottle.
    \item After 10 minutes of fermentation, data was exported to a USB and the \ce{CO2} gas sensor was removed. Observations were made as the reaction took place.
    \item The \ce{CO2} gas sensor was wiped to remove condensation and fanned to blow away residual \ce{CO2}. The bottle was thoroughly rinsed and also fanned to remove any yeast-glucose solution and residual \ce{CO2}.
    \item The 1\,\si{L} beaker was filled with more distilled water if there was not enough to half-submerge the bottle.
    \item Steps 2 to 14 were repeated until 2 trials of data were collected for 0.10\,\si{g}, 0.25\,\si{g}, 0.50\,\si{g}, and 1.00\,\si{g} of glucose.
\end{enumerate}

\section{Data Collection and Processing}
Listed below are general qualitative observations that were common with all trials:
\begin{itemize}[topsep=\parskip, noitemsep]
    \item When all the yeast was dissolved, the solution turned into a consistency similar to oat-milk with a light brownish colour
    \item The glucose dissolved very quickly when stirred
    \item When the solution was dumped after each trial, a smell of ethanol was noted
\end{itemize}

\medskip

The following table displays qualitative observations that were observed for different masses of glucose added. A corresponding concentration is calculated for each mass:
\begin{table}[H]
\centering
\caption{Observations and concentrations for different masses of glucose added}
\label{table:4}
\begin{tabularx}{\textwidth} {
    | >{\hsize=.3\hsize \linewidth=\hsize \raggedright\arraybackslash}X
    | >{\hsize=.3\hsize \linewidth=\hsize \raggedright\arraybackslash}X
    | >{\hsize=.4\hsize \linewidth=\hsize \raggedright\arraybackslash}X |}
    \hline
    \textbf{Mass /\ce{g} $\pm\textbf{0.01}$} & \textbf{Concentration /\si{M}} & \textbf{Observations} \\
    \hline
    0.10 & $0.0056 \pm 0.0006$ & Enough condensation to cover top of bottle, solution was very consistent and smooth \\
    \hline
    0.25 & $0.0139 \pm 0.0006$ & A lot of condensation, considerably more than 0.1\,\si{g}, some condensation started to drip back into the solution \\
    \hline
    0.50 & $0.0278 \pm 0.0007$ & When mixing solution, some white foam appeared on top, a good amount of condensation was formed \\
    \hline
    1.00 & $0.0555 \pm 0.0008$ & Less condensation was observed than smaller concentrations, traces of very fine precipitate left at the bottom depsite stirring with a stir bar \\
    \hline
\end{tabularx}
\end{table}

A sample calculation for the concentration uncertainty using the basic method is provided below for mass 0.50\,\si{g}. The molar mass of glucose is taken to be 180.18\,\si{g.mol^{-1}} \parencite{data-booklet}:
\begin{align*}
    c &= \frac{n}{v} = \frac{m/M}{v} \\
    &= \frac{0.5/180.18}{0.1} = 0.02775\,\si{M} \\  %% TODO better equations
    \Delta c &= c\left(\frac{\Delta m}{m} + \frac{\Delta v}{v}\right) = 0.02775\left(\frac{0.01}{0.5} + \frac{0.0005}{0.1}\right) \\
    &= 0.000694\,\si{M} \\
    \therefore\,c &= (0.0278 \pm 0.0007)\,\si{M}
\end{align*}

\newpage

%% TODO reaction velocity vs rate of reaction
The raw data collected were concentration-time graphs generated by the Vernier \ce{CO2} gas sensor. In total, 2 sets of data were collected for each concentration. As an example, the first trial for 0.5\,\si{g} of glucose added is shown below. The initial reaction velocity was taken to be the slope of the line of best fit for the whole scatter plot because the reaction has not reached equilibrium. A full list of raw data can be found in the appendix.  %% TODO specify which appendix, full "list" -> "list" not the right word to use
\begin{figure}[H]
    \renewcommand{\figurename}{Graph}
    \setcounter{figure}{1}
    \centering
    \includegraphics{figures/graph_02.png}
    \vspace*{-12pt}
    \caption{Carbon dioxide concentration plotted against time for 0.5\,\si{g} of glucose added}
    \label{fig:graph2}
\end{figure}

Additionally, the graphs below displays all the raw data with the line of best fits ommited due to clutter:
\begin{figure}[H]
    \renewcommand{\figurename}{Graph}
    \setcounter{figure}{2}
    \centering
    \includegraphics{figures/graph_03.png}
    \vspace*{-12pt}
    \caption{Concentration-time graphs for grams of glucose added in the first trial}
    \label{fig:graph3}
\end{figure}

\begin{figure}[H]
    \renewcommand{\figurename}{Graph}
    \setcounter{figure}{3}
    \centering
    \includegraphics{figures/graph_04.png}
    \vspace*{-12pt}
    \caption{Concentration-time graphs for grams of glucose added in the second trial}
    \label{fig:graph4}
\end{figure}

The following table displays the equation for the line of best fit and $R^2$ values for each concentration:
\begin{table}[H]
\centering
\caption{Observations and concentrations for different masses of glucose added. Uncertainty for average slope was calculated using the half range method.}
\label{table:5}
\begin{tabularx}{\textwidth} {
    | >{\hsize=.26\hsize \linewidth=\hsize \raggedright\arraybackslash}X
    | >{\hsize=.27\hsize \linewidth=\hsize \raggedright\arraybackslash}X
    | >{\hsize=.27\hsize \linewidth=\hsize \raggedright\arraybackslash}X
    | >{\hsize=.2\hsize \linewidth=\hsize \raggedright\arraybackslash}X |}
    \hline
    \textbf{Glucose concentration /\si{M}} & \textbf{Line of best fit equation and $R^2$ for trial 1} & \textbf{Line of best fit equation and $R^2$ for trial 2} & \textbf{Average slope (initial reation velocity) /\si{ppm.s^{-1}}} \\
    \hline
    $0.0056 \pm 0.0006$ & $y=35.324x, R^2=0.9962$ & $y=81.96x, R^2=0.9936$ & $59 \pm 20$ \\
    \hline
    $0.0139 \pm 0.0006$ & $y=59.095x, R^2=0.9805$ & $y=85.884x, R^2=0.9979$ & $72 \pm 10$ \\
    \hline
    $0.0278 \pm 0.0007$ & $y=59.426x, R^2=0.9954$ & $y=82.694x, R^2=0.9886$ & $71 \pm 10$ \\
    \hline
    $0.0555 \pm 0.0008$ & $y=36.929x, R^2=0.9789$ & $y=68.013x, R^2=0.9648$ & $52 \pm 20$ \\
    \hline
\end{tabularx}
\end{table}

\newpage

With the processed data in tablex, reaction velocity can be plotted against glucose concentration to obtain the Michaelis Menten model. Doing a curve fit in excel reveals $V_{max} \approx 64.1\,\si{ppm.s^{-1}}$ and $K_m \approx 0.001\,\si{M}$
\begin{figure}[H]
    \renewcommand{\figurename}{Graph}
    \setcounter{figure}{4}
    \centering
    \includegraphics{figures/graph_05.png}
    \vspace*{-12pt}
    \caption{The Michaelis Menten model derived from a curve of best fit plotted against experimental values}
    \label{fig:graph5}
\end{figure}

\section{Analysis}
Firstly, all concentration-time graphs were normalized such that the carbon dioxide concentration reading at $t=0$ is 0, making interpreting them easier. This is because the initial concentration reading in the room is the same for all trials, and this experiment is more concerned with reaction velocity (slope) rather than the final \ce{CO2} concentration.

\medskip

For the most part, all concentration-time graphs followed a linear trend for the first 10 minutes of the reaction. This implies a zero order reaction, meaning at this stage, the initial reaction velocity is independent of the glucose concentration. However, none of the graphs exhibited the end behaviour as the reaction approaches equilibrium, because it was only allowed to react for 10 minutes. Thus, the fermentation process of yeast was a slower reaction than proposed in the hypothesis.

\medskip

The processed data in Graph (3) somewhat fit the general trend of the Michaelis Menten model. The theoretical curve fitted to the data is valid because it crosses through all the vertical error bars. From the calculated values of $V_{max}$ and $K_m$, several conclusions can be made about the fermentation process of yeast. A very low value of $K_m$ suggests that enzymes have a very high affinity for the substrate \parencite{biochem}, and which makes sense because according to Equation (15), a small $K_m$ implies a large denominator, meaning the concentration of enzyme substrate complex is significantly larger.

\medskip

Additionally, the Michaelis Menten model is consistent with the raw data. Starting from around 0.005\,\ce{M}, the initial reaction velocity starts to become independent of the substrate concentration. This is consistent with the concentration-time graphs because they start off as 0 order reaction, and their initial reaction velocity seems to reside in the range where the Michaelis Menten curve starts to become asymptotic. Theoretically, the fermentation could start off as a first order reaction if the glucose concentration was small enough, since the initial reaction velocity increases linearly with changes of concentration about the origin.

\section{Evaluation}
The largest potential source of error was deduced to be the temperature of the water, because it was difficult to keep constant, especially in between trials when the warm water bath was refilled. The temperature reading on the magnetic stirring hotplate wasn't the same as the water, meaning it was adjusted a few times, leaving room for a range of solution temperatures, which increases or decreases the rate at which carbon dioxide gas was produced, if the temperature was higher or lower, respectively.

\medskip

Observing the Michaelis Menten plot, the reaction velocity should increase as substrate concentration increases, not decrease, as observed with the plotted data. This implies that the reaction velocity should have been higher when it was not. The two major assumed reasons for this major discrepancy are listed as follows.

\medskip

First of all, adding yeast and glucose to the bottle when the insides are wet causes some of the substances to stick to the sides. Although the bottle was swirled with water after this happened, it's likely that not all the yeast ended up reacting to all the glucose. Secondly, the time between experiments could have been too short. Usually the first experiment of the day yields the most accurate results, but it's possible that not enough time was given carbon dioxide in the \ce{CO2} gas sensor to completely diffuse out, decreasing the effectiveness in the instrument.

\medskip

Additionally, escaping carbon dioxide could also be a contributing factor, but it should have a very minute impact because the bottle was sealed off by an airtight rubber portion of the \ce{CO2} gas sensor.

\medskip
These skewed results may also reflect the raw data obtained from the plotted concentration-time graphs. For example, the reaction velocity in Graph (b) from Appendix 2 seems to change at about 350 seconds. This sudden change in slope is believed to be incoherent stirring, since the dent at the bottom of the Nalgene bottle makes it hard for the stir bar to reach every corner of the bottle. Lastly, Graph (d) from both Appendix 2 and 3, which both happen to come from a solution with 1g of glucose added, seems to vaguely emulate the behavior of a first order reaction. A potential explanation for this could be the time it takes for a larger amount of glucose to dissolve, giving an underestimate for the initial reaction velocity. Unsurprisingly, these three graphs also have the smallest $R^2$ values with regards to their lines of best fit.

\section{Suggestions for Future Study}
Throughout all of the trials, the reaction was stopped before reaching equilibrium, so the end behaviour of the reaction could not be studied. By giving the yeast more time for fermentation and potentially collecting the produced carbon dioxide gas through water displacement methods could allow for further analysis on theoretical and actual yield.

\section{Conclusion}
Although the experiment did not match up perfectly with the theory, the results still made sense as a whole for a sufficient level of analysis. Evidently, this experiment models an extremely downscaled version of the many aspects involved in baking, thus presenting itself as more of a study into fermentation and enzyme kinetics rather than a practical guide to baking the best loaf of bread. In conclusion, the explanation of yeast producing carbon dioxide to cause bread to rise is widely accepted by many, but also an understatement, because it often unacknowledges the complex and fascinating chemical mechanisms involved.

\nocite{*}

\newpage

\printbibliography

\newpage

\captionsetup[table]{name=Appendix}
\setcounter{table}{0}

\section{Appendix}
All the data exported from the LabQuest is compiled into a large table below:
\begin{table}[H]
\centering
\caption{Full table of collected raw data}
\label{table:appendix1}
\begin{tabularx}{\textwidth} {| X | X | X | X | X | X | X | X | X |}
    \hline
    \textbf{Time /s} & \multicolumn{8}{l |}{\textbf{Concentration of \ce{CO2} over time given mass of glucose added /ppm}} \\
    \cline{2-9}
    & \multicolumn{2}{l |}{\textbf{0.10\,\ce{g}}} & \multicolumn{2}{l |}{\textbf{0.25\,\ce{g}}} &\multicolumn{2}{l |}{\textbf{0.50\,\ce{g}}} & \multicolumn{2}{l |}{\textbf{1.00\,\ce{g}}} \\
    \cline{2-9}
    & \textbf{Trial 1} & \textbf{Trial 2} & \textbf{Trial 1} & \textbf{Trial 2} & \textbf{Trial 1} & \textbf{Trial 2} & \textbf{Trial 1} & \textbf{Trial 2} \\
    \hline
    0 & 231 & 481 & 261 & 135 & 198 & 135 & 135 & 231 \\
    \hline
    20 & 826 & 4395 & 1451 & 2985 & 731 & 3267 & 511 & 700 \\
    \hline
    40 & 1358 & 7650 & 2516 & 5396 & 1263 & 6117 & 856 & 1013 \\
    \hline
    60 & 1858 & 10124 & 3550 & 7620 & 2140 & 8778 & 1139 & 1419 \\
    \hline
    80 & 2390 & 12096 & 4362 & 9373 & 3237 & 10530 & 1482 & 1953 \\
    \hline
    100 & 2985 & 13161 & 5146 & 11032 & 4082 & 12003 & 1890 & 2609 \\
    \hline
    120 & 3613 & 14162 & 5991 & 12535 & 5083 & 13035 & 2296 & 3550 \\
    \hline
    140 & 4145 & 15415 & 6899 & 13725 & 6304 & 13630 & 2766 & 4551 \\
    \hline
    160 & 4738 & 16542 & 7870 & 14820 & 7713 & 14257 & 3517 & 5678 \\
    \hline
    180 & 5396 & 17796 & 8904 & 15917 & 8997 & 15259 & 4112 & 7025 \\
    \hline
    200 & 6210 & 18827 & 9779 & 17262 & 10281 & 16073 & 4801 & 8276 \\
    \hline
    220 & 7181 & 20643 & 10656 & 19329 & 11721 & 17357 & 5615 & 9748 \\
    \hline
    240 & 7900 & 21803 & 11501 & 20580 & 13035 & 18921 & 6397 & 11219 \\
    \hline
    260 & 8621 & 23211 & 12283 & 21677 & 14132 & 20237 & 7275 & 12659 \\
    \hline
    280 & 9466 & 24807 & 13161 & 22741 & 15352 & 21238 & 8120 & 14069 \\
    \hline
    300 & 10248 & 25528 & 14194 & 24181 & 16636 & 22272 & 9060 & 15478 \\
    \hline
    320 & 11032 & 26686 & 15196 & 26405 & 18608 & 23682 & 10281 & 17168 \\
    \hline
    340 & 11814 & 27752 & 16167 & 28502 & 20205 & 24933 & 11375 & 19516 \\
    \hline
    360 & 12629 & 28847 & 17826 & 29818 & 21271 & 26905 & 12316 & 21145 \\
    \hline
    380 & 13287 & 29881 & 19705 & 31258 & 22335 & 28721 & 13254 & 22867 \\
    \hline
    400 & 13943 & 32135 & 21301 & 32572 & 23243 & 30256 & 14225 & 24464 \\
    \hline
    420 & 14664 & 33575 & 22930 & 34607 & 24275 & 32009 & 15039 & 26562 \\
    \hline
    440 & 15385 & 34796 & 24870 & 36768 & 25558 & 34264 & 16010 & 28940 \\
    \hline
    460 & 16041 & 36205 & 26937 & 38364 & 27719 & 36329 & 17262 & 31384 \\
    \hline
    480 & 17012 & 38677 & 29036 & 41464 & 29943 & 38301 & 18797 & 34170 \\
    \hline
    500 & 18139 & 41464 & 31258 & 44094 & 31664 & 39555 & 20174 & 36674 \\
    \hline
    520 & 19423 & 42561 & 33419 & 46003 & 32604 & 41716 & 21271 & 39366 \\
    \hline
    540 & 20519 & 44094 & 35578 & 47602 & 33699 & 45315 & 22146 & 42028 \\
    \hline
    560 & 21301 & 45973 & 37802 & 49322 & 34889 & 49072 & 23180 & 44815 \\
    \hline
    580 & 21803 & 48414 & 40117 & 51201 & 36110 & 54083 & 24181 & 47602 \\
    \hline
    600 & 22711 & 49292 & 42717 & 53143 & 37270 & 59999 & 25278 & 50106 \\
    \hline
\end{tabularx}
\end{table}

\end{document}
