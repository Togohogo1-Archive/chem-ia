\documentclass[12pt]{article}

\usepackage[utf8]{inputenc}
\usepackage[style=apa]{biblatex}
\usepackage[margin=1in]{geometry}
% \usepackage[indent=0.5in]{parskip}  % remember to add comment about why this needed (it also has some issues)
\usepackage{hyperref}
\usepackage{indentfirst}
\usepackage{mhchem}
\usepackage{enumitem}

\addbibresource{references.bib}
\setlength{\bibitemsep}{12pt}
\setlength{\parindent}{0in}
% \setlength{\parskip}{12pt}
\renewcommand{\baselinestretch}{1.15}  % 1.15 spacing

\title{\textbf{The Relationship between Glucose Concentration and the Rate of Carbon Dioxide Production during the Fermentation Process of Yeast}}
\author{}
\date{}

% https://chem-space.com/search
\begin{document}

\pagenumbering{gobble}
\maketitle
\newpage
\pagenumbering{arabic}

\section{Introduction}
I have always enjoyed cooking as a mostly self-taught hobby. Having experimented with multiple aspects of cooking throughout my life eventually led me to explore baking, where I was fascinated by timelapse footages of bread rising in the oven. As such, I was inspired to make my own bread, following basic recipes online with a surface-level understanding of the chemistry involving the release of carbon dioxide from yeast fermentation. However, the realm of baking is very large and not limited to bread only, as there exist many types of doughs for pizza, puff pastry, cookies, etc. One notable difference between these different types of dough is the amount of sugar used, which led me to form my research question: \textbf{``The Relationship between Glucose Concentration and the Rate of Carbon Dioxide Production during the Fermentation Process of Yeast''}.

\section{Background}

\subsection{What Is Yeast}
Yeast are single celled, eukaryotic organisms of the fungus kingdom. Like many living organisms, yeast requires food to survive, namely be converting carbohydrates or sugar sources to energy through either aerobic or anaerobic respiration \parencite{ref}. There exists many types of yeast. The one used in this internal assessment are part of the \emph{Saccharomyces cerevisiae} strain, which can be categorized into brewer's yeast (beer making) and baker's yeast (leavening agent) \parencite{ref}.

\medskip

This internal assessment focuses on the use of active dry yeast, which consists of pellets of live yeast cells coated with a layer of dehydrated cells, \parencite{ref} allowing it to stay dormant to give it a longer shelf life. Hance, active dry yeast must be activated through rehydration in lukewarm water from 37$\,^{\circ}$C to 43$\,^{\circ}$C (optimal fermentation conditions) for 5 to 10 minutes in a process called proofing which also ensures that the yeast is dissolved in the water \parencite{ref}.

\subsection{Aerobic and Anaerobic Respiration in Yeast}
It is well known that yeast reacts with glucose to produce ethanol and carbon dioxide. To understand the details of how this works, it is crucial to understand how yeast undergoes both aerobic and anaerobic respiration.

\medskip

Firstly, aerobic respiration is much more efficient and optimal than anaerobic respiration because of the significantly more ATP (adenosine triphosphate, energy carrying molecule) produced. In the presence of oxygen, yeast reacts with glucose to produce carbon dioxide and water \parencite{ref}:
\begin{equation}
    \ce{C6H12O6 + 6O2 ->[cellular respiration] 6CO2 + 6H2O}
\end{equation}
Going into the details of aerobic respiration its three steps:
\begin{enumerate}[topsep=0pt, noitemsep]
    \item Glycolysis
    \item Tricarboxylic acid cycle
    \item Oxidative phosphorylation (where most of the APT is produced)
\end{enumerate}

\medskip

However, the focus of this internal assessment is on anaerobic respiration of yeast because the amount of oxygen was limited given the environment that the experiment was carried out in. The only similarity between aerobic and anaerobic respiration is glycolysis, which prompts to explore this step in more detail \parencite{ref}.

\medskip

The first step of glycolysis converts glucose into fructose 1,6-biphosphate, outlined in the figure below:

[fig- Conversion of glucose to fructose 1,6-biphosphate. Intermediate steps are ommited. Image created by moi]

Glucose goes through a series of reactions catalyzed by multiple enzymes, which consumes 2 ATP in the process. ATP turns to ADP (adenosine diphosphate) when a phosphate group gets transferred from the ATP to the sugars that glucose gets broken down into, eventually leading to the formation of fructose 1,6-biphosphate.


\section{Kinetics}
Lorem ipsum dolor sit amet, consectetur adipiscing elit. Aenean pellentesque eros eget egestas mollis. Duis non mauris imperdiet, tristique ipsum vel, vestibulum mi. Nunc eget urna arcu. Morbi neque leo, accumsan at euismod ac, tempor eget purus. Curabitur a varius velit. Curabitur non imperdiet ipsum. Fusce porta odio eu nibh mattis molestie. Maecenas ut dictum mauris, consequat euismod lacus.

\subsection{Basic Kinetics}
Lorem ipsum dolor sit amet, consectetur adipiscing elit. Aenean pellentesque eros eget egestas mollis. Duis non mauris imperdiet, tristique ipsum vel, vestibulum mi. Nunc eget urna arcu. Morbi neque leo, accumsan at euismod ac, tempor eget purus. Curabitur a varius velit. Curabitur non imperdiet ipsum. Fusce porta odio eu nibh mattis molestie. Maecenas ut dictum mauris, consequat euismod lacus.

\subsection{Michaelis Menten Enzyme Kinetics}
Lorem ipsum dolor sit amet, consectetur adipiscing elit. Aenean pellentesque eros eget egestas mollis. Duis non mauris imperdiet, tristique ipsum vel, vestibulum mi. Nunc eget urna arcu. Morbi neque leo, accumsan at euismod ac, tempor eget purus. Curabitur a varius velit. Curabitur non imperdiet ipsum. Fusce porta odio eu nibh mattis molestie. Maecenas ut dictum mauris, consequat euismod lacus.

\nocite{*}

\newpage

\printbibliography

\end{document}