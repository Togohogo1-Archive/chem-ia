\documentclass[12pt]{article}

\usepackage[style=apa]{biblatex}
\usepackage[margin=1in]{geometry}
\usepackage[utf8]{inputenc}
\usepackage[indent=0.5in]{parskip}  % remember to add comment about why this needed
\usepackage{hyperref}
\usepackage{indentfirst}
\usepackage{soul}

\addbibresource{references.bib}
\setlength{\bibitemsep}{12pt}
\setlength{\parskip}{12pt}
\renewcommand{\baselinestretch}{1.15}  % 1.15 spacing

\title{\textbf{The Relationship between Glucose Concentration and the Rate of Carbon Dioxide Production during the Fermentation Process of Yeast}}
\author{}
\date{}

% https://chem-space.com/search
\begin{document}

\pagenumbering{gobble}
\maketitle
\newpage
\pagenumbering{arabic}

\section{Introduction}
I have always enjoyed cooking as a mostly self-taught hobby. Having experimented with multiple aspects of cooking throughout my life eventually led me to explore baking, where I was fascinated by timelapse footages of bread rising in the oven. As such, I was inspired to make my own bread, following basic recipes online with a surface-level understanding of the chemistry involving the release of carbon dioxide from yeast fermentation. However, the realm of baking is very large and not limited to bread only, as there exist many types of doughs for pizza, puff pastry, cookies, etc. One notable difference between these different types of dough is the amount of sugar used, which led me to form my research question: \textbf{``The Relationship between Glucose Concentration and the Rate of Carbon Dioxide Production during the Fermentation Process of Yeast''}.

\section{Background}

\subsection{What Is Yeast}
Yeast are single celled, eukaryotic organisms of the fungus kingdom. Like many living organisms, yeast requires food to survive, namely be converting carbohydrates or sugar sources to energy through either aerobic or anaerobic respiration \parencite{}.

There exists many types of yeast. The one used in this internal assessment are part of the \emph{Saccharomyces cerevisiae} strain, which can be categorized into brewer's yeast (beer making) and baker's yeast (leavening agent). \parencite{}

This internal assessment will focus on the use of active dry yeast

\subsection{Aerobic and Anaerobic Respiration in Yeast}
Lorem ipsum dolor sit amet, consectetur adipiscing elit. Aenean pellentesque eros eget egestas mollis. Duis non mauris imperdiet, tristique ipsum vel, vestibulum mi. Nunc eget urna arcu. Morbi neque leo, accumsan at euismod ac, tempor eget purus. Curabitur a varius velit. Curabitur non imperdiet ipsum. Fusce porta odio eu nibh mattis molestie. Maecenas ut dictum mauris, consequat euismod lacus.

\section{Kinetics}
Lorem ipsum dolor sit amet, consectetur adipiscing elit. Aenean pellentesque eros eget egestas mollis. Duis non mauris imperdiet, tristique ipsum vel, vestibulum mi. Nunc eget urna arcu. Morbi neque leo, accumsan at euismod ac, tempor eget purus. Curabitur a varius velit. Curabitur non imperdiet ipsum. Fusce porta odio eu nibh mattis molestie. Maecenas ut dictum mauris, consequat euismod lacus.

\subsection{Basic Kinetics}
Lorem ipsum dolor sit amet, consectetur adipiscing elit. Aenean pellentesque eros eget egestas mollis. Duis non mauris imperdiet, tristique ipsum vel, vestibulum mi. Nunc eget urna arcu. Morbi neque leo, accumsan at euismod ac, tempor eget purus. Curabitur a varius velit. Curabitur non imperdiet ipsum. Fusce porta odio eu nibh mattis molestie. Maecenas ut dictum mauris, consequat euismod lacus.

\subsection{Michaelis Menten Enzyme Kinetics}
Lorem ipsum dolor sit amet, consectetur adipiscing elit. Aenean pellentesque eros eget egestas mollis. Duis non mauris imperdiet, tristique ipsum vel, vestibulum mi. Nunc eget urna arcu. Morbi neque leo, accumsan at euismod ac, tempor eget purus. Curabitur a varius velit. Curabitur non imperdiet ipsum. Fusce porta odio eu nibh mattis molestie. Maecenas ut dictum mauris, consequat euismod lacus.

\nocite{*}

\newpage

\printbibliography

\end{document}